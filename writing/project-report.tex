% article example for classicthesis.sty
\documentclass[10pt,a4paper]{article} % KOMA-Script article scrartcl
\usepackage{lipsum}
\usepackage{url}
\usepackage[nochapters]{classicthesis} % nochapters

\begin{document}
	\pagestyle{plain}
	\title{\rmfamily\normalfont\spacedallcaps{Text Mining Twitter: Colombia 2018 Elections}}
	\author{\spacedlowsmallcaps{juan sebastián garcía rodriguez} \\ \spacedlowsmallcaps{travis dunlop}}
	\date{} % no date
	
	\maketitle
	
	\begin{abstract}
		\noindent Between May and June of 2018 the people of Colombia will vote for their next president.  As with any modern election, people are using Twitter, the social media platform, to support candidates they like, discredit the others, and debate who should win.  Twitter provides a massive open forum to create dialogue across the country.  Some groups take advantage of this by creating \textit{social bots}, which automatically post political tweets to in an attempt to sway votors \cite{swaine_2018}.  In our project, we use the text mining skills gained in this course to assess the influence of these bots.  We hope to answer two questions: how many bots are there and who are they supporting?
		
	\end{abstract}
	
	\section{Introduction}
		In order to answer these two questions we split the project into two sections:  bot detection and sentiment analysis.  For bot detection we leverage an existing algorithm and use semi-supervised learning to extrapolate it's findings.  With sentiment analysis, we hand-labeled a portion of the tweets and use those to help classify the rest into positive, negative, or neutral. \\
		
		\noindent The data we use is a set of [[INSERT NUMBER HERE]] recently collected tweets that mention any of the Colombian candidates.  By the end of the analysis we report what percent of the tweets support which candidate, and how many likely bots we were able to detect.

	\section{Bot Detection} 
		Dectecting social bots is a complicated phenomenon, because they usually aim to not be found.  This results in a game of cat-and-mouse - researchers attempting more sophisticated classification strategies, and bot-makers emulating increasingly human-like behavior.  One technique developed to identify bots is to use a \textit{honeypot}.  In this case, the honeypot is a set of researcher-created Twitter users who tweet mostly nonsense.  The users that interact with them are likely to be bots -- exploiting their desire to engage with many users.  Of course, some real users just happen to message or follow one of these bots.  In order to parse out which are the real, the researchers use unsupervised learning to cluster the data into groups.  They find that some of the clusters seem more real than others.  This particular strategy was developed and implemented by Lee et. all \cite{}.  Once this group of bots are identified, features are extracted to compare them with other users.  The researchers use things like frequency of tweets, number of followers, time between posting, ratio of tweets to retweets, even the length of the username as features in the model.  Another group from the Indiana University Network Science Institute has made this model available as a API \cite{DavisVFFM16}.  Unfortunately, the API has limits on access and so, we label some users and then use a model to extrapolate our results to label other users. \\
		
		\noindent Specifically
	\section{Sentiment Analysis}
	
	\section{Conclusion}
	
	% bib stuff
	\nocite{*}
	\addtocontents{toc}{\protect\vspace{\beforebibskip}}
	\addcontentsline{toc}{section}{\refname}
	\bibliographystyle{plain}
	\bibliography{project-report}
\end{document}